\begin{table}[h!]

\centering

\caption{Matriz de diseño metodológico.}
\label{tab:matriz-metodologico}

\resizebox{16cm}{!} {

\begin{tabular}{|c|cccc|}
\hline
\textbf{\begin{tabular}[c]{@{}c@{}}Pregunta de \\ investigación\end{tabular}} & \multicolumn{4}{c|}{\begin{tabular}[c]{@{}c@{}}¿Sera que con la aplicación del método de redes neuronales, se obtendrá \\ información adecuada con mayor aproximación a la situación \\ económica-financiera de la institución financiera analizada?\end{tabular}} \\ \hline
\textbf{\begin{tabular}[c]{@{}c@{}}Objetivo \\ general\end{tabular}} & \multicolumn{4}{c|}{\begin{tabular}[c]{@{}c@{}}Proporcionar información financiera adecuada con mayor aproximación a la de \\ decisiones situación económica-financiera observada, mediante la determinación \\ de pronósticos de estados financieros por el método de redes neuronales artificiales.\end{tabular}} \\ \hline
\textbf{\begin{tabular}[c]{@{}c@{}}Objetivos \\ específicos\end{tabular}} & \multicolumn{1}{c|}{\begin{tabular}[c]{@{}c@{}}Diagnosticar la \\ situación actual \\ del sistema financiero \\ de Bolivia.\end{tabular}} & \multicolumn{1}{c|}{\begin{tabular}[c]{@{}c@{}}Definir la \\ arquitectura y \\ entrenamiento \\ del modelo de red \\ de neuronas \\ artificiales.\end{tabular}} & \multicolumn{1}{c|}{\begin{tabular}[c]{@{}c@{}}Proyectar y simular \\ los estados \\ financieros.\end{tabular}} & \begin{tabular}[c]{@{}c@{}}Evaluar \\ los datos \\ proyectados-simulados \\ respecto a los datos \\ observados.\end{tabular} \\ \hline
\textbf{\begin{tabular}[c]{@{}c@{}}Unidad de \\ análisis\end{tabular}} & \multicolumn{1}{c|}{CAMEL} & \multicolumn{1}{c|}{\begin{tabular}[c]{@{}c@{}}RED\\ NEURONAL\end{tabular}} & \multicolumn{1}{c|}{\begin{tabular}[c]{@{}c@{}}RED\\ NEURONAL\end{tabular}} & CAMEL \\ \hline
\textbf{\begin{tabular}[c]{@{}c@{}}Tipo de \\ fuente\end{tabular}} & \multicolumn{1}{c|}{Secundaria} & \multicolumn{1}{c|}{Secundaria} & \multicolumn{1}{c|}{Secundaria} & Secundaria \\ \hline
\textbf{\begin{tabular}[c]{@{}c@{}}Técnica de \\ recolección\end{tabular}} & \multicolumn{1}{c|}{\begin{tabular}[c]{@{}c@{}}Revisión \\ bibliográfica\end{tabular}} & \multicolumn{1}{c|}{\begin{tabular}[c]{@{}c@{}}Revisión \\ bibliográfica\end{tabular}} & \multicolumn{1}{c|}{\begin{tabular}[c]{@{}c@{}}Revisión \\ bibliográfica\end{tabular}} & \begin{tabular}[c]{@{}c@{}}Revisión \\ bibliográfica\end{tabular} \\ \hline
\textbf{\begin{tabular}[c]{@{}c@{}}Información \\ necesaria\end{tabular}} & \multicolumn{1}{c|}{\begin{tabular}[c]{@{}c@{}}Estados Financieros \\ del sistema financiero \\ de Bolivia.\end{tabular}} & \multicolumn{1}{c|}{\begin{tabular}[c]{@{}c@{}}Elementos de la \\ red neuronal, \\ numero de neuronas, \\ funciones de activación \\ y funciones de coste.\end{tabular}} & \multicolumn{1}{c|}{\begin{tabular}[c]{@{}c@{}}Estados financieros \\ estructurados en \\ forma vectores.\end{tabular}} & \begin{tabular}[c]{@{}c@{}}Estados \\ financieros \\ proyectados\end{tabular} \\ \hline
\end{tabular}

}

\end{table}
